% aps.tex
%
% This file is root file for an example annual progress seminar report
% Created by Amey Karkare (18 August 2007)
%
% It is provided without warranty on an AS IS basis.


%=====================================================================
% Document Style
%=====================================================================
% APS format: There is no prescribed format by IIT Bombay 

% \documentclass[twoside]{iitbaps}  %to print on both sides of paper
\documentclass[oneside]{iitbaps}  %to print on one side of paper

% To include optional packages, us5e the \usepackage command.
% For e.g.:
\usepackage{epsfig}
\usepackage[breaklinks]{hyperref}
\usepackage[all]{hypcap}
\usepackage{amsmath,amssymb,amsbsy,amsfonts}
\usepackage[numbers,sort]{natbib}
\bibliographystyle{abbrvnat}
\DeclareMathOperator*{\argmin}{arg\,min}
\usepackage{booktabs}
\usepackage{multirow}
\usepackage{inputenc}
%\usepackage[utf8]{inputenc}
%\usepackage[T1]{fontenc}
\usepackage{longtable}
%\usepackage{appendix}
% \usepackage{chemformula}
% \usepackage{chemist}
\DeclareMathOperator*{\argmax}{arg\,max}
\usepackage{wrapfig}
\usepackage{textcomp}            %% for making \mu non italic
\usepackage{subcaption}          %% for subfigures
\usepackage{bm}
\usepackage{graphicx}
%\usepackage{caption}
\usepackage{setspace}
\usepackage{mathtools}
\usepackage{color}
\usepackage{multirow}
\usepackage{soul} % for striking text
\usepackage{bm}
  %\renewcommand*\thesubsubsection{%
  %\arabic{chapter}.\arabic{section}.\arabic{subsection}.\arabic{subsubsection}%

%   \usepackage{textgreek}

%%%%%%%%%%%%%%%%%%%%%%%%%%%%%%%%%%%%%%%%%%%%%%%%%%%%%%%%%%%%%%
\newcommand*{\defeq}{\stackrel{\text{def}}{=}} % for defined equal to
%%%%%%%%%%%%%%%%%%%%%%%%%%%%%%%%%%%%%%%%%%%%%%%%%%%%%%%%%%%%%%
\newcommand{\thickhat}[1]{\mathbf{\hat{\text{$#1$}}}} % hat bar and tilde thick
\newcommand{\thickbar}[1]{\mathbf{\bar{\text{$#1$}}}}
\newcommand{\thicktilde}[1]{\mathbf{\tilde{\text{$#1$}}}}

%%%%%%%%%%%%%%%%%%%%%%%%%%%%%%%%%%%%%%%%%%%%%%%%%%%%%%%%%%%%%%
\newtheorem{myaxiom}{Axiom} % for axiom

\newcommand\independent{\protect\mathpalette{\protect\independenT}{\perp}} % for independence symbol
\def\independenT#1#2{\mathrel{\rlap{$#1#2$}\mkern2mu{#1#2}}}               % for independence symbol
%%%%%%%%%%%%%%%%%%%%%%%%%%

%%%%%%%%%%%%%%%%%%%%%%%%%%
%=======================================================================
% End of Preamble, start of document
%

\begin{document}

% Choose your bibliography style
% plain is the basic style, others include ieeetr, siam, asm, etc

% \bibliographystyle{wileyj}


% \include{defs}                 % Common definitions used throughout the thesis
%============================================================================
% prelude.tex
%   - titlepage
%   - dedication (optional)
%   - table of contents, list of tables and list of figures
%   - nomenclature
%   - abstract
%============================================================================


\clearpage%
\pagenumbering{roman}  % This makes the page numbers Roman (i, ii, etc)

%--------------------------------------------------------------------%
% TITLE PAGE
%   - define \title{} \author{} \date{}

\title{YOUR TITLE}
\author{YOUR NAME}
\date{24\textsuperscript{th} October, 2020}

%  - Roll number, required for title page, approval sheet, and
%    certificate of course work 
\rollnum{12345678} 

%   - The default report type is preliminary report.
%     For any other type, use  \reporttype{}
\reporttype{First Annual Progress Report}
% \aps is a shorthand for \reporttype{progress seminar report}
%\aps
 

%\iitbdegree{Doctor of Philosophy}
%   - The default department is ``Unknown Department''
%     The department can be changed using the command \department{}
\department{{\bf Department of ABC Engineering}}
%\department{Department of Civil Engineering}

%    - Set the guide's name
\setguide{Prof.~ ABC \& Dr.~DEF\\ IIT Bombay Supervisors}
%    - Set the coguide's name (if you have one)
%\setcoguide{Prof.~Jayadipta Ghosh \\ IIT Bombay Supervisor}

%    - Set the coguide's name (if you have one)
\setcoguide{Dr.~GHI\\ Monash University Supervisor}

%   - once the above are defined, use \maketitle to generate the titlepage
\maketitle

%---------------------------------------------------------------------%
%ACCEPTANCE CERTIFICATE
%\makeapproval
\clearpage
\vspace{15mm}
\begin{center}
{\fontsize{16pt}{0}
\selectfont {\bf{Acceptance Certificate}}}
\end{center}

\vspace{5mm}
\begin{center}
{\bf DEPARTMENT OF ABC ENGINEERING\\
INDIAN INSTITUTE OF TECHNOLOGY BOMBAY}
\end{center}
The Third Annual Progress Seminar Report entitled {\bf ``YOUR THESIS TITLE''}, submitted by 
{\bf Mr. YOUR NAME} is accepted for evaluation. \\

\vspace{5mm}
\noindent \hspace{0.40 in} Date \hspace{4.10 in} Supervisor
\vspace{5mm} \\
% \noindent \hspace{0.5 in} 16$^{th}$ June, 2014.\\
% \vspace{2mm} 
\noindent \rule{0.2\textwidth}{1pt} \hspace{3.1 in} \rule{0.3\textwidth}{1pt}

\hspace{4.025 in} {\bf  Prof.\ ABC}
\vspace{5mm} \\
\noindent \rule{0.2\textwidth}{1pt} \hspace{3.1 in} \rule{0.3\textwidth}{1pt}

\hspace{4.025 in} {\bf  Prof.\ DEF}
\vspace{5mm} \\

\noindent \rule{0.2\textwidth}{1pt} \hspace{3.1 in} \rule{0.3\textwidth}{1pt}

\hspace{4.025 in} {\bf  Prof.\ GHI}
% \hspace{4.29 in} Associate Professor
% 
% \hspace{4.35 in} Civil Engineering
% 
% \hspace{4.09 in} Depertment, IIT Bombay


\clearpage
%--------------------------------------------------------------------%
% ABSTRACT
% \begin{abstract}
% \input{abstract}
% \end{abstract}

%--------------------------------------------------------------------%
% CONTENTS, TABLES, FIGURES
% \tableofcontents
% \listoftables
% \listoffigures
%--------------------------------------------------------------------%
%\input{listofsymbols}
% NOMENCLATURE
% \begin{nomenclature}
% \begin{description}
% \item{\makebox[0.75in][l]{$C_1$}} Constant 1
% 
% \item{\makebox[0.75in][l]{$V$}}    Voltage 
% 
% \item{\makebox[0.75in][l]{\$}}     US Dollars
% \end{description}
% \end{nomenclature}


%--------------------------------------------------------------------%
%  Single counter for theorems and theorem-like environments:
\newtheorem{theorem}{Theorem}[chapter]
\newtheorem{assertion}[theorem]{Assertion}
\newtheorem{claim}[theorem]{Claim}
\newtheorem{conjecture}[theorem]{Conjecture}
\newtheorem{corollary}[theorem]{Corollary}
\newtheorem{definition}[theorem]{Definition}
\newtheorem{example}[theorem]{Example}
\newtheorem{figger}[theorem]{Figure}
\newtheorem{lemma}[theorem]{Lemma}
\newtheorem{prop}[theorem]{Proposition}
\newtheorem{remark}[theorem]{Remark}

%--------------------------------------------------------------------%
% Make the page numbers Arabic (1, 2, etc)
\cleardoublepage%
\pagenumbering{arabic}
           % Title page, abstract, table of contents, etc
\tableofcontents
 \chapter{Chapter 1}              % Chapter 1: 
 
% \include{chapter2}		% Chapter 2:
% \include{fig_chp2}
% \include{tab_chp2}
% \include{chapter3}		% Chapter 3: 
% \include{fig_chp3}
% \include{table_chp3}
% \include{chapter4}		% Chapter 4: 
% \include{table_chp4}
% \include{chapter5}		% Chapter 5: 
%                 
% \include{fig_chp5} 
%  \include{table_chp5}
%\include{chapter6}
%\include{figure_chp6}
%\include{table_chp6}
%\include{chp_conclusions}      % Chapter: 
%\include{fig_conclusions}

%--------------------------------------------------------------------%
% APPENDIX
%  Appendices, if any, must precede the cited literatures.
%  Appendices shall be numbered in Roman Capitals (e.g. Appendix IV)
%\appendix                      
%\appendixpage
%\addappheadtotoc
%\begin{appendices}
%\input{appendix_1}        % Appendices if any..  
%\end{appendices}
%--------------------------------------------------------------------%

%   conclusions chapter.

\bibliography{references}              % Make the bibliography


% \clearpage
% \begingroup
% \renewcommand\bibname{List of publication}
% \input{publication.bbl}
% \endgroup

% ACKNOWLEDGMENTS
%   For APS, acknowledgements can be the first item after title page
%      OR
%   the very last item.
%
% \begin{acknowledgments}
% \input{acknowledgment}
% \end{acknowledgments}


\end{document}
